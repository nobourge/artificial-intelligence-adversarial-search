\documentclass{article}
\usepackage{hyperref}
\usepackage{graphicx}
\usepackage{listings}

\title{INFO-F-311: Artificial Intelligence - Project 2: Recherche adversariale}
\author{Bourgeois Noé}
\date{2023 October 29}

\begin{document}

\maketitle

\tableofcontents

\newpage
\section{Introduction}
This report outlines the application of adversarial search techniques in graph search problems. For references, please refer to the project instructions.

\section{Environment Setup}
\subsection{Rust}
Instructions for setting up Rust environment.

\subsection{Poetry}
Instructions for setting up Poetry and Python environment.

\subsection{Tests}
Instructions for running automated tests using pytest.

\newpage
\section{Results}
\subsection{Minimax}
Discussion and analysis of results using Minimax.

\subsection{Alpha-beta pruning}
Discussion and analysis of results using Alpha-beta pruning.

\subsection{Expectimax}
Discussion and analysis of results using Expectimax.

\newpage
\section{Heuristics Development}
Discussion of evaluation functions used in multi-agent adversarial scenarios.

\newpage
\section{Comparative Analysis}
Comparative analysis of states expanded using different algorithms on custom maps.

\newpage
\section{ChatGPT Usage}
\subsection{Data Dump}
Updated data dump information for this project.

\subsection{Filtering Output}
Description of how ChatGPT's output was filtered for this project.

\newpage
\section{Bug Reporting in Laser Learning Environment}
Instructions for reporting bugs in LLE.

\newpage
\section{References}
Updated list of references.

\end{document}
